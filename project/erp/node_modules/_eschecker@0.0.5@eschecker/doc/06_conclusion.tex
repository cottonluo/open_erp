\section{Conclusion}\label{sec:conclusion}
The tool support for JavaScript is especially small compared to its popularity. Developers need to relay on manual testing or unit tests for revealing programming errors. The implemented type checker provides a tool that is capable of inferring the types and catch a variety of errors through type checking. The evaluation shows that the analysis is precise and sound for most of the scenarios and sometime provides better results than competitive tools. Therefore, the tool can provide valuable feedback. 

But the evaluation also shows that the presented algorithm has its limitations. First, the precision decreases for very dynamic code, for instance, when objects are dynamically manipulated. The precision can be improved if type annotations are used in these cases to substitute type inference. A second limitation is the inability to access properties of potential absent values as the analysis is not path-sensitive and therefore, the value is potentially absent in all branches. Path sensitivity is also required to support type-specific branches, a technique often used to emulate function overloading. 

Further, the set of supported features is not sufficient to analyze real-word projects. The implementation is still missing elementary features, such as classes, prototyping or modules. These features are all essential and needed before the tool is useful. Supporting the features defined in ECMAScript 6 and in the upcoming ECMAScript 7 standard requires a tremendous amount of additional work that exceeds the scope of a project thesis by a multitude. But this project thesis shows that precise inference results can be achieved for a majority of JavaScript that, combined with type checking, provides a valuable and immediate feedback to programmers. 